\documentclass{itmo}

\setlist[itemize]{label=---}
\setlist[enumerate]{label=\arabic*)}

\usepackage{dcolumn}
\newcolumntype{d}[1]{D{.}{.}{#1}}

\begin{document}

\makereporttitle{Программной инженерии и компьютерной техники (ФПИ и КТ)}
	{лабораторной работе №3}
	{Хранение и алгоритмы сжатия данных}
	{Исследование формата хранения данных в СУБД SQLite}
	{}
	{Постнов~С.~А./P4135}
	{Бабаянц~А.~А.}

\maketableofcontents

\chapter{Теоретическая часть}

\section{Формат хранения данных в SQLite}

\texttt{SQLite}~---~реляционная СУБД, хранящая данные в едином файле.
Основные особенности:
\begin{enumerate}
	\item файл состоит из страниц фиксированного размера (512~--~65536 байт);
	\item заголовок файла (100 байт) содержит метаданные: сигнатуру <<\texttt{SQLite format 3}>>, размер страницы, версию и  кодировку;
	\item данные организованы в B--деревьях: \texttt{table b-tree} для таблиц (rowid + payload), \texttt{index b-tree} для индексов.
	\item переменная длина целых чисел (1--9 байт) для сжатия;
	\item данные строки включают header (varint с размерами колонок) и body (\texttt{NULL}, \texttt{INTEGER}, \texttt{REAL}, \texttt{TEXT}, \texttt{BLOB});
	\item \texttt{sqlite\_master}~---~системная таблица на странице 1 с метаданными о всех объектах.
\end{enumerate}

\chapter{Практическая часть}

В лабораторной работе реализованы две программы на \texttt{C} для исследования формата хранения данных в \texttt{SQLite}:
\begin{enumerate}
	\item \texttt{sqlite\_writer.c}~---~генератор минимального \texttt{SQLite} файла с таблицей \texttt{mytable(id INTEGER PRIMARY KEY, name TEXT)} и двумя записями;
	\item \texttt{sqlite\_reader.c}~---~парсер \texttt{SQLite} файла с данными, выводящий содержимое указанной таблицы.
\end{enumerate}

\section{Реализация программ}

\begin{itemize}
	\item Чтение/запись заголовка файла и страниц.
	\item Кодирование/декодирование varint.
	\item Парсинг record format с поддержкой \texttt{NULL}, \texttt{INTEGER}, \texttt{TEXT}.
	\item Построение B--дерево страниц для таблиц.
\end{itemize}

\section{Результат работы}

\includelisting
	{sqlite3_output.txt}
	{\texttt{sqlite3 mydb.sqlite "SELECT rowid, name FROM mytable;"}}

\includelisting
	{my_output.txt}
	{\texttt{./sqlite\_reader mydb.sqlite mytable}}

\chapter{Вывод}

По результатам исследования можно сделать следующий вывод:
\begin{enumerate}
	\item формат хранения данных в \texttt{SQLite} эффективен для хранения структурированных данных с использованием varint и record format для сжатия;
	\item программы корректно работают и совместимы с \texttt{sqlite3}.
\end{enumerate}

\end{document}

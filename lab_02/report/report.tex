\documentclass{itmo}

\setlist[itemize]{label=---}
\setlist[enumerate]{label=\arabic*)}

\usepackage{dcolumn}
\newcolumntype{d}[1]{D{.}{.}{#1}}


\begin{document}

\makereporttitle{Программной инженерии и компьютерной техники (ФПИ и КТ)}
	{лабораторной работе №2}
	{Хранение и алгоритмы сжатия данных}
	{Сравнение форматов хранения данных Parquet и ORC}
	{}
	{Постнов~С.~А./P4135}
	{Бабаянц~А.~А.}

\maketableofcontents

\chapter{Теоретическая часть}

\section{Apache Parquet}

\texttt{Apache Parquet}~---~колоночный формат хранения данных, оптимизированный для аналитических запросов.
Основные особенности:
\begin{itemize}
	\item Колоночное хранение данных для эффективного сжатия и быстрого чтения.
	\item Встроенная схема данных с типами.
	\item Поддержка вложенных структур данных.
	\item Оптимизация для больших данных и аналитических запросов.
	\item Использование алгоритмов сжатия Snappy, Gzip, LZ4.
\end{itemize}

\section{Apache ORC}

\texttt{Apache ORC} (\textit{Optimized Row Columnar})~---~высокопроизводительный колоночный формат хранения данных.
Основные особенности:
\begin{itemize}
	\item Гибридное хранение (строки и колонки).
	\item Встроенные индексы для быстрого поиска.
	\item Агрегированная статистика на уровне полос (stripes).
	\item Эффективное сжатие с использованием Zlib, Snappy, LZ4.
	\item Оптимизация для чтения больших объемов данных.
\end{itemize}

\chapter{Практическая часть}

В лабораторной работе использовались следующие датасеты для сравнения форматов:
\begin{enumerate}
	\item \texttt{trade\_data.csv}~---~данные торгов на бирже ($\sim$266~МБ);
	\item \texttt{market\_orders.csv}~---~данные заказов на маркетплейсе ($\sim$684~МБ);
	\item \texttt{tweets.csv}~---~данные постов пользователей в социальной сети ($\sim$3.9~ГБ).
\end{enumerate}

\includelistingpretty
	{trade_data.csv}
	{}
	{Пример данных из файла \texttt{trade\_data.csv}}

\includelistingpretty
	{market_orders.csv}
	{}
	{Пример данных из файла \texttt{market\_orders.csv}}
	
\includelistingpretty
	{tweets.csv}
	{}
	{Пример данных из файла \texttt{tweets.csv}}
	
\section{Выбранное преобразование данных}

Для сравнения форматов были выполнены следующие операции:
\begin{enumerate}
	\item Чтение исходных csv файлов.
	\item Применение преобразований данных (добавление колонки с суммой длин всех полей).
	\item Сохранение данных в форматах \texttt{Parquet} и \texttt{ORC}.
	\item Измерение времени чтения сжатых файлов.
	\item Вычисление коэффициентов сжатия.
\end{enumerate}

Все измерения проводились с очисткой кэша \texttt{Spark} для обеспечения точности результатов.

\section{Результаты сравнения}

В таблице \ref{tab:results} представлены результаты сравнения форматов хранения данных.
\begin{table}[ht!]
	\centering
	\caption{Результаты сравнения форматов хранения данных}
	\label{tab:results}
	\begin{tabularx}{\textwidth}{|c|X|X|X|X|X|X|}
	\hline
	Датасет & Размер csv, МБ & Размер \texttt{Parquet}, МБ & Размер \texttt{ORC}, МБ & Время чтения \texttt{Parquet}~,~с & Время чтения \texttt{ORC}, с \\
	\hline
	\texttt{trade\_data} & 265.53 & 56.92 & 45.12 & 0.36 & 0.11 \\
	\hline
	\texttt{market\_orders} & 684.14 & 200.70 & 182.68 & 0.08 & 0.05 \\
	\hline
	\texttt{tweets} & 3997.58 & 2182.62 & 1439.69 & 0.08 & 0.09 \\
	\hline
	\end{tabularx}
\end{table}

Коэффициенты сжатия для различных форматов представлены в таблице \ref{tab:compression}.

\begin{table}[ht!]
	\centering
	\caption{Коэффициенты сжатия}
	\label{tab:compression}
	\begin{tabularx}{\textwidth}{|c|X|X|}
	\hline
	Датасет & Сжатие \texttt{Parquet} & Сжатие \texttt{ORC} \\
	\hline
	\texttt{trade\_data} & 4.7x & 5.9x \\
	\hline
	\texttt{market\_orders} & 3.4x & 3.7x \\
	\hline
	\texttt{tweets} & 1.8x & 2.8x \\
	\hline
	\end{tabularx}
\end{table}

\clearpage
На рисунке \ref{img:comparison} представлено сравнение размеров файлов, коэффициентов сжатия и скорости чтения для всех исследуемых датасетов.
\includeimage
	{comparison}
	{f}
	{H}
	{\textwidth}
	{Сравнение форматов хранения данных}

\chapter{Вывод}

По результатам проведенного сравнения можно сделать следующий вывод:
\begin{enumerate}
	\item \texttt{ORC} превосходит \texttt{Parquet} по сжатию~---~коэффициент сжатия ORC на $\sim$20~--~30\% выше для большинства датасетов;
	\item \texttt{ORC} быстрее читается~---~время чтения \texttt{ORC} файлов в $\sim$2~--~4 раза меньше по сравнению с \texttt{Parquet} для большинства случаев;
	\item оба формата значительно превосходят csv~---~размер файлов уменьшается в $\sim$1.8~--~5.9 раз, а скорость чтения увеличивается в $\sim$5~--~50 раз.
\end{enumerate}

\end{document}

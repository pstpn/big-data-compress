\documentclass{itmo}

\setlist[itemize]{label=---}
\setlist[enumerate]{label=\arabic*)}

\usepackage{dcolumn}
\newcolumntype{d}[1]{D{.}{.}{#1}}
\usepackage{tabularx}

\begin{document}

\makereporttitle{}
	{лабораторной работе №4}
	{Хранение и алгоритмы сжатия данных}
	{Сравнение алгоритмов сжатия данных}
	{}
	{Постнов~С.~А./P4135}
	{Бабаянц~А.~А./преподаватель}
	{}

\maketableofcontents

\chapter{Теоретическая часть}

\section{Алгоритм сжатия Хаффмана}

Алгоритм Хаффмана — это метод энтропийного сжатия данных без потерь, разработанный Дэвидом Хаффманом в 1952 году.
Он основан на построении префиксного кода, где более частые символы получают более короткие коды.

Основные шаги:
\begin{enumerate}
	\item Подсчёт частоты каждого символа в данных.
	\item Построение дерева Хаффмана: символы с наименьшей частотой объединяются в узлы до получения одного корневого узла.
	\item Присвоение кодов: левым рёбрам — 0, правым — 1.
	\item Кодирование данных с использованием полученных кодов.
\end{enumerate}

Декодирование происходит с помощью того же дерева.

\section{Библиотека zlib}

\texttt{zlib} — это библиотека для сжатия данных, реализующая алгоритм DEFLATE, который сочетает LZ77 и код Хаффмана.
DEFLATE использует словарное сжатие (LZ77) для поиска повторяющихся последовательностей и динамическое кодирование Хаффмана для сжатия.

Особенности:
\begin{enumerate}
	\item Эффективно сжимает текстовые и бинарные данные.
	\item Поддерживает уровни сжатия от 0 (без сжатия) до 9 (максимальное сжатие).
	\item Широко используется в форматах ZIP, PNG, HTTP.
\end{enumerate}

\chapter{Практическая часть}

В лабораторной работе реализована программа на \texttt{C++} для сравнения эффективности алгоритма сжатия Хаффмана (собственная реализация) и библиотеки \texttt{zlib}.
Программа тестирует обе реализации алгоритма на различных типах файлов: текстовом (\texttt{file.txt}), PDF (\texttt{file.pdf}), DOC (\texttt{file.doc}), ZIP (\texttt{file.zip}) и BMP (\texttt{file.bmp}).

\section{Реализация программ}

\begin{itemize}
	\item Реализация алгоритма Хаффмана: построение дерева частот, генерация кодов, кодирование и декодирование.
	\item Замеры времени кодирования и декодирования при помощи реализованного алгоритма и \texttt{zlib}.
	\item Расчёт коэффициента сжатия как отношение размера сжатых данных к оригинальным.
	\item Вывод результатов в CSV-файл для дальнейшего анализа.
\end{itemize}

\section{Результат работы}

Результаты сравнения представлены в таблице~\ref{tab:results} и на графике~\ref{img:all_plots}.

\begin{table}[h]
\centering
\caption{Результаты сравнения алгоритмов сжатия}
\label{tab:results}
\begin{tabularx}{\textwidth}{|l|l|X|X|X|}
	\hline
	Алгоритм & Файл & Коэффициент сжатия  & Время сжатия (мс) & Время разжатия (мс) \\
	\hline
	Huffman & \texttt{data/file.txt} & 2.608696 & 0.174500 & 0.095292 \\
	\hline
	zlib & \texttt{data/file.txt }& 6.666667 & 0.101958 & 0.064083 \\
	\hline
	Huffman & \texttt{data/file.pdf} & 0.999635 & 1420.730750 & 764.234750 \\
	\hline
	zlib & \texttt{data/file.pdf }& 1.042540 & 151.265916 & 114.895542 \\
	\hline
	Huffman & \texttt{data/file.doc} & 1.572061 & 18.582125 & 8.246333 \\
	\hline
	zlib & \texttt{data/file.doc} & 2.738191 & 2.636458 & 1.003500 \\
	\hline
	Huffman & \texttt{data/file.zip} & 0.998123 & 275.354208 & 147.683166 \\
	\hline
	zlib & \texttt{data/file.zip} & 1.000645 & 24.997625 & 21.995875 \\
	\hline
	Huffman & \texttt{data/file.bmp} & 1.049384 & 73.750625 & 38.998917 \\
	\hline
	zlib & \texttt{data/file.bmp} & 1.181160 & 8.448833 & 5.912958 \\
	\hline
\end{tabularx}
\end{table}

\includeimage
	{all_plots}
	{f}
	{H}
	{\textwidth}
	{Результаты сравнения реализаций алгоритмов сжатия}

\chapter{Вывод}

Исходя из результатов сравнения, можно сделать следующие выводы:
\begin{enumerate}
	\item \texttt{zlib} показывает лучший коэффициент сжатия для большинства файлов (текст, PDF, DOC, BMP), за исключением уже сжатых (ZIP), где оба алгоритма близки.
	\item Время кодирования и декодирования у \texttt{zlib} значительно быстрее, особенно для больших файлов (PDF).
	\item Алгоритм Хаффмана эффективен для простых случаев, но \texttt{zlib} превосходит по скорости и степени сжатия благодаря комбинации LZ77 и динамического кодирования.
\end{enumerate}

\end{document}
